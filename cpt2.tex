\chapter{SELECTするってどんなこと}


\section{SELECTの性質}

\subsection{フィルタとしてのSELECT}

\subsection{数学の関数をSELECTで書く}

SELECTとはなにかを一言で表すとしたら、それは数学でいう関数です。

たとえば、$y=x~2+1$という関数を考えてみます。この式は2次関数で、2次元直交座標系では、このようにあらわされます。
SELECtが関数であれば、これをそのまま表現することができるはずです。

ここで、xaxisというテーブルを考えます。このテーブルは、直交座標系でx軸の上にある数値をとります。その名の通り、xaxisに含まれるレコードは、数値で、属性名がxであるとします。

テーブルxaxisのすべてのレコードは、重複なく、x軸上の数値を表します。つまり、属性xはxaxisのプライマリキーです。このような条件の下、xの値に対応したyの値のすべては、以下のSELECT文で求めることができます。

\begin{verbatim}
SELECT x*x+1 AS y FROM xaxis
\end{verbatim}

このSELECTの実行結果は、xaxisに含まれるすべてのレコードそれぞれのxの値に対応するyの値の集合になります。

範囲を設定することもできます。x軸上で、[1,10]という区間を競ってしたときのグラフは、以下のようになります。
この区間のyを求める関数は、SELECTでは以下のように書くことができます。

\begin{verbatim}
SELECT x*x+1 AS y FROM xaxis WHERE x>=1 AND x<=10
\end{verbatim}

このように、SELECTとは関数です。関数はxとyの間の対応(写像関係)を表します。

\subsection{SELECTと副作用}

SELECTは、数学で言う関数と同じものであることを説明しました。では、関数の性質として、どのようなものがあるでしょうか。

$y=f8x)$という関数を考えてみます。xとyの関係を

\subsection{SELECTの結果}

\subsection{集合とテーブルの違い}

SELECTは、数学で言うところの関数です。

\section{SELECTとテーブル}

\subsection{なぜサブクエリという概念があるのか}

\subsection{命題と解空間}

\section{SELECTと第一級関数}


\subsection{第一級関数型言語}