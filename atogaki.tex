\chapter{あとがき}

\section*{メイ・カートミル(仮名)とマージ・ニコルス(仮名)の秋葉原での会話}

\begin{quotation}
\noindent
{\bf メイ}「マージさん、最近アズレンやってるねすか?」 \\
{\bf マージ}「うん、これがアズレンの吹雪」 \\
{\bf メイ}「艦これよりもえっちぃですね」 \\
{\bf マージ}「それ以上いけない」 \\
\end{quotation}


\section*{後書き}

お兄ちゃん、SQLを覚えたいなら、お友達に頭をストライクしてもらって手続き型言語のことを全て忘れるんだ。

まず、今回は表紙をハッピーブックねばーる号さんに寄稿していただきました。ハッピーブックねばーる号さん、ありがとうございます。
タイトルの暗に、SELECT破戒録SQLというのがあったというのは本当の話ですよ。ざわ…………ざわ…………

というわけで、インフラエンジニアの毒舌な妹(@infra\_imouto)です.これを書いている数日、寒かったり暖かったり、お兄ちゃん、お体の具合はいかがですか?

そういえばネットワーク以外の本は私がメイン筆者になるんだなとおもいながら、あきらめわるく技術書典直前の数日を本気出して過ごしています。あとは参考文献をビルドしてアップロードして、ざわ…………ざわ…………して、っと。

SQLの本というよりも、テーブルとSELECTの説明に終始した感があります。とはいえ、手続き型言語との違いを感じるには必要なところとおもい、今回はこのテーマに絞りました

残り時間の都合で書き切れなかったのが、制御構造も条件分岐もいらない、という内容の章です。
SQLでは、ループや条件分岐を使わなずに、フィルタリングやデータ加工をすることができます。そういう部分の解説ができなかったのが心残りと言いつつ、入稿まで4時間を切っている今現在なのですよ。ざわ…………ざわ…………

そのあたりは、次のSQLの本でやろうと言うことになったのですが、またSQLの本出すか。うん、いい世だそうじゃないか、というわけで、次回あったらまたお会いしましょうね。

お兄ちゃん、Haskellわかるって自慢してるのに、なんでSQLがわからないの?

\begin{flushright}
2019年4月14日 \\
インフラエンジニアの毒な妹 \\
\end{flushright}

本書の表紙を担当してくださったハッピーブックねばーる号さんに謝辞を送ります。まさに悪魔的表紙というか。私もそっちの影響受けていますね。

今回は、共著者として、図版と数学的なレクチャーを担当しました。サークル主催ことありすゆうです。

後書きなので裏話を暴露してしまうと、本書は、締切りラスト数日で、内容の半分以上を書き換えました。今回は日光企画様の特急プランのお世話になる予定です。

リレーショナルデータベースは、リレーショナル演算を行うためのデータベースです。そして、リレーショナル演算は、集合論や命題論理にそのベースがあります。SELECTはどうみても集合論的な意味で関数ですが、このあたりはもっと突っ込んでいきたいところ。サークル内でかききれなかった部分を次のヒントして出そう、という話もしています。

次は7月にイベントがあり、8月がコミケ、そして多分10月に次の技術書典、そんなイベントでまた皆様とお会いできますように。

\begin{flushright}
2019年4月14日 \\
ありす ゆう
\end{flushright}


%\newpage
% ここまでで160ページ鳴ったのでブランクなし
% 1ページブランクを入れる

%\thispagestyle{empty}
%\mbox{}
%\newpage
%\clearpage

% PICOはノンブルがいる
%\thispagestyle{empty}
\mbox{}
\newpage
\clearpage


% PICOはノンブルがいる
%\thispagestyle{empty}

\vspace*{\fill}
\begin{tabular}{ll} \toprule
筆者 & インフラエンジニアの毒舌な妹 ありす ゆう\\
発行 & AliceSystem \\
連絡先 & aliceyou@alicesystem.net \\
URL & http://aliceyou.air-nifty.com/onesan/ \\
初版発行日 & 2019年4月14日 \\
印刷所 & 日光企画  \\ \bottomrule
\end{tabular}