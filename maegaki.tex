\section*{謝辞}
\begin{center}
この本を読んでくださる方に \\
気力をくれる友人に \\
大切な人に \\
感謝と本書をささげます
\end{center}

\section*{前書き}

今回は、初のSQL本をおおくりします。何やらゆるそうなタイトルですが、その実は……なんて脅してはいけないですね。
SQLというのは、手続き型言語と比べたとき、考え方が根本的に違う言語です。

集合に対しての操作を意識しないといけないので、頭を完全に切り赤得る必要があります。
むしろHaskellから遅延評価を抜いたような感じ、といういささか乱暴な説明で済ませたくなるところです。

本書では、そんな風に、頭を切替えるノ比必要な、SQLという考え方の初歩についてまとめました。もし、本書で、SQLが分からなかった理由がクリアされれば、それに勝る喜びはありません。

\section*{本書の内容}

\paragraph{第一章}

まず、SQLの本の選び方について説明します。

内容的な言及もありますが、SQLという言語の本と、リレーショナルデータベース実装の本を、どのように区別して、必要な本を見つけるか、という点についての説明が中心です。

\paragraph{第二章}

この章では、テーブルとは空間であること、アトリビュートが座標であること、という考え方を説明します。

リレーショナルデータベースの、テーブルとアトリビュートとレコードについて説明をしています。テーブルはスプレッドシートではないこと、レコードはスプレッドシートの行ではないこと、アトリビュートはスプレッドシートの列で無いことについて説明しています。

また、テーブルの中のレコードに順番が無いことについても説明しています。


\paragraph{第三章}

SQLにおいて、SELECTとはなにをしているかについての説明をする章です。
SELECTがフィルタであること、ラムダ式のようなものでもあること、関数と等価であることについて説明をしています。

また、そこから、サブクエリを使用できる根拠について説明をします。

その一方で、、SQLは関数型言語ではありません。
その理由として、関数型言語のどんな要素が足りないかについっても、記述しています。

\paragraph{第四章}

SQLにおけるNULLの考え方と、三値論理について説明しています。

NULLには知らない、適用できない、という二つの意味があることを説明し、その考えを導入して、二値論理を拡張した三値論理について説明をします。

\paragraph{第五章}

UNIONとJOINに着いて節をします。

最初に、二つの結果を合成するUNIONに着いて説明します。
次に、入力テーブル側で、二つのテーブルを混ぜるJOINの考え方について記述しています。

交差結合、内部結合の説明を行い、NULLを許容する結合として、左右の外部結合の説明を行います。

\section*{免責事項}
本書に書いてあることは、筆者知識のレベルでまとめたものです。ですが、内容が正しいとは言い切れません。初版でも改訂版でも相当やらかしています。また、学校のレポート、業務などのコードを書く際に、本書の内容を信じて書いて損害が生じても、筆者にその責任はありません。

くれぐれも、自己責任と十分な検証の上、ご利用ください。

\section*{表紙イラスト}
