\chapter{UNIONとJOIN}

\section{UNIONすること}

UNIONは、複数のSELECT文を繋ぐのに使われます。では、UNIONは何を繋いでいるのでしょうか。

ここまで出何度も説明したように、SELECT文と、その結果であるテーブルは等価でする。そのため、SELECTをつなぐということは、その結果のなるテーブルを繋ぐということと同じです。つまり、複数ある集合を混ぜて、ひとつの集合を作っているということです。

\subsection{テーブルを繋ぐ}

UNIONでテーブルを繋ぐと、つまり、集合を混ぜる、というのはどういうことでしょうか。
集合を混ぜる、というのは、それぞれの集合がもつ要素をひとつの集合としてあつめ、共通の座標系で場所を表現するということです。

つまり、各要素は、同じ座標系で場所を表現できる必要があります。これをSQLの言葉で表現すると、アトリビュートが共通であることが必要ということです。同じアトリビュートをもつテーブルを結果として返すSELECTふたつをつなぐと、以下のように書くことができます。

この例は、テーブルxaxisからxの値が10より大きい、もしくは-10撚りも小さいレコードを全て集めたテーブルを得るUNIONになります。

\begin{verbatim}
SELECT x FROM xaxis WHERE x > 10
  UNION
SELECT x FROM xaxis WHERE x < -19
\end{verbatim}

ただし、アトリビュートは、型がおなじであれば、名前はSELECTの中で別名を付けてやればUNIONできます。

以下の例では、xという整数のアトリビュートをもつレコードのテーブルxaxisと、yという実数のアトリビュートをもつレコードのテーブルyaxisから、全ての値を持ってきて、二つの結果をまぜた、整数型のyというアトリビュートをもつレコードのテーブルを返すSQLです

\begin{verbatim}
SELECT x as y FROM xaxis
  UNION
SELECT int(y) as y FROyM yaxis ;
\end{verbatim}


\subsection{UNIONとレコードの重複}

UNIONで結果テーブルをあわせてひとつにつすとき、内容が重複したらどのようになるのでしょうか。前の章の数式に相当するSELECT文を、UNIONををつかって書いてみましょう。

\begin{verbatim}
SELECT x as y FROM xaxis
WHERE x > 0
  UNION
SELECT 1 as y FROM xaxis
WHERE x = 0
  UNION
SELECT -x as y FROM xaxis
WHERE x < 0
\end{verbatim}

このような書方をした場合、結果テーブルのレコードの要素yは、重複が生じる可能性があります。このとき、重複が排除され、要素の数の合計は、必ずしもそれぞれのSELECTb文の結果との数の合計とは一致しません。

\section{JOINするってどういうこと?}

JOINも、テーブルを混ぜ、新しいテーブル変わりはありません。ですが、UNIONとは、決定的な違いがあります。それは、レコードのアトリビュートが異なる、つまり座標軸が異なるテーブルをまぜる操作であるということです。そして、混ぜた結果の可能性をスベtレ網羅することでもあります。

JOINは、元になるテーブルを組み合わせて、新しいテーブルを作ることです。そうやって作られたテーブルは名前を持ちません。
また、寿命も、そのSELECT文のなかで使われる間だけとなります。そのため、サブクエリと同様に、ラムダ式のようなものと考えることもできます。

\subsection{交差結合}

一番簡単なJOINをしてみることにしましょう。たとえば、名前、という属性だけをもつレコードのテーブルと、所属、という属性だけをもつレコードのテーブルを考えます。

この二つのテーブルをJOINすると、どうなるでしょうか。名前、という座標軸をもつテーブルと、所属、という属性をもつテーブルをまぜるというのは、名前と所属という、ふたうの座標軸をもった、新しいテーブルを作るということです。そのテーブルに所属するレコードは、当然ながら、元になった二つのテーブルのレコードが持っていたアトリビュートを全て持っていります。

では、その新しいテーブルの要素は、いくつ存在するのでしょうか。それは、名前テーブルの要素の数と、所属テーブルの要素の数をかけたものになります。
なぜなら、名前テーブルのあるひとつの要素は、所属テーブルの要素が取っている値の全てと対応している可能性があります。
逆に、所属テーブルの要素は、斜めテーブルの要素が取っている値の全てと対応している可能性があります。

ここでは、JOINしたふたつのテーブルの要素の組み合わせで、取り得る可能性全てを網羅しています。この組み合わせかたを、交差結合(Cross Join)と呼びます。数学では、直積といいます。

\subsection{交差結合をどこで使うのか}

交差結合は、要素として取り得る者全てを網羅します。そのため、あり得ない組み合わせも一緒に網羅することになります。

たとえば、動物名前を著わすレコードのテーブルと動物の種類を著わすレコードのテーブルを交差結合したとしましょう。動物の種類の座標軸で計った、種類が鳥である、という場所に、犬や猫やパンダが来ることになります。
ですが、この組早生があり会えないことは、データが持つ意味で人間が判断することであって、交差結合では、全ての可能性が網羅されます。

そんな交差結合は、自己相関結合という形で使われることがあります。あるテーブルの要素に大小の区別を付けることができるとき、上から何番目かを調べる場合などに使います。
ここでは、sizeという要素の値が、そのテーブルの中で何番目に大きいかを出してみましょう。ある要素が何番目に大きいかは、自分自身と同じかそれより大きい者がいくつあるかを数え上げれば求めることができます。

まず、自分自身と交差結合をします。そうすると、ある要素に注目したとき、その要素と全ての要素を組み合わせた要素をもつレコードのテーブルができあがります。
そこから、結合した要素の値が、比較対象の要素同じか大きいものだけを抜き出して、テーブルを作ることにしましょう。

\begin{verbatim}
SELECT size.t1 , size.t2 FROM
  table as t1 CROSS JOIN table as t2
WHERE size.ti <= size.t2 ;
\end{verbatim}

このSELECT文は、size.t1とsize.t2という二つの座標軸をもつテーブルと等価です。その要素は、size.t1とsize.t2を比較したとき、size.t1と同じか、size.t1野方が小さい要素だけ画存在するテーブルとなります。

次に、そのテーブルをsize.t1でグループ化して、集計関数でグループに纏められた要素数を数えれば、それは、あるsize.t1の値が何番目に大きいかを表す数値となります。

テーブルとSELECTは等価なので、先程の交差結合をしたSELECTをサブクエリとして、順番をシャットオーケーするSELECTの入力として与えます。そうsると、全体で、以下のSELECT文となります。

\begin{verbatim}
SELECT size.t1 , count(*) FROM
  SELECT size.t1 , size.t2 FROM
    table as t1 CROSS JOIN table as t2
  WHERE size.ti <= size.t2
GROUP BY size.t1 ;
\end{verbatim}

\subsection{INNER JOIN}

次に、内部結合(INNER JOIN)に着いて説明をします。内部結合は、取り得る要素の可能性から、結合してできるテーブルに残る要素としてどれを残すかを定義することができる結合です。

交差結合は、二つのテーブルを混ぜたときの、取り得る可能性全てを網羅するものでした。これは、意味的にあり得ないものも含めて、全ての可能性を網羅するものでもあります。

では、その全ての可能性から、あり得ないと判断するものを除く方法はあるのでしょうか。たとえば、動物の名前と種別コードという二つの属性をもつレコードのテーブルanimalsと、種別コードと、動物の種別という二つの属性をもつレコードのテーブルkindsがあるとします。

この時、ふたつのテーブルの種別コードが同じ型と意味を持っているなら、それを基準に、あり得ない可能性を排除することができます。それは、それぞれの種別コードが一致する組み合わせだけを、要素として取り出す、ということです。

INNER JOINという結合子でテーブルを二つ結合します。そして、結合したテーブルの全ての可能性から、どの組み合わせを残したテーブルにするのか、という条件を、ONの後に書きます。ここでは、animalsテーブルのkindidの値と、kindosテーブルのidの値が一致するものだけを、要素として残します。

\begin{verbatim}
SELECT name.animals , name.kinds FROM
  animals INNER JOIN kinds ON kindid.animals = id.kinds ; 
\end{verbatim}

内部結合で気をつけるところは、この例では、animalsの全レコードの、kindidにない値をとるkindsのレコードとの組み合わせ、kindsの全レコードのidになり値のkindidをもつanimalsのレコードとの組み合わせは、現れないということです。ONのあとに書いた条件と一致する組み合わせから作られるレコードだけが、結合で作られ留テーブルの要素となります。

\subsection{JOINした結果のレコードのアトリビュートの区別}

ここまで、交差結合と内部結合という、ふたつのJOINに着いて説明しました。では、JOINしたとき、そのそれぞれのテーブルに由来するアトリビュートは、どのように区別されるのでしょうか。

JOINでは、元になるテーブルのアトリビュートは、一致していても違っていても構いません。では、名前の一致している跡取りユーとがそれぞれのテーブルにあるときに、そのアトリバイ-とはどう区別されるのでしょうか。

アトリビュートの区別は、由来するテーブルの名前を付加することによって行われます。本来、アトリビュートの正式な名前は、その名前の後に、コンマでテーブル名を付けたものです。そのため、ふたつのテーブルをJOINする場合、完全にアトリビュートの正式名が一致することはありません。

例外は自己相関結合の場合で、このときは、あらかじめ、ASでテーブル名をつけ、そのSELKECTのなかでは、別のテーブルに見えるようにしておきます。
そうすることで、JOINしたあとの名無しのテーブルでは、アトリビュートの名前の後にデー手ベースの別名がつき、区別することが可能となります。

\subsection{左外部結合と右外部結合}

内部結合は、アトリビュートを組み合わせてできた新しいレコードのなかから、新しいテーブルに入れる要素を選ぶ基準を、明確に定義するものでした。

では、その条件を判定するときに、知らない、つまり、判定できない、ということを許容するとどうなるでしょうか。SQLでは、レコードのアトリビュートで、そのアトリビュートについては適用できるものがない、ということをあらわす値として、NULLが存在します。このNULLを許容した結合が、左外部結合と、右外部結合です。

内部結合では、結合の可能性があるふたつのレコードの間で条件を設定しています。そして、その条件が成立する組み合わせのみを取ることは、ここまで説明したとおりです。

ここで、ONの条件で適用できる組み合わせがない場合は、どちらか片方のテーブルのレコードを残し、それに対する組み合わせでは、適用できるものが無かった、という情報を組み合わせます。この、適用できなかった、という情報は、NULLであらわします。
つまり、JOINされた結果のテーブルは、アトリビュートにNULLをもつレコードが含まれる可能性があります。

なぜ、左外部結合、右外部結合と呼ぶかは、結合子JOINの左側にあるテーブルと右側にあるテーブルとのレコードで、比較対象であるアトリビュートで条件に一致しない場合、結合結果のテーブルで、結合子の左に書いたテーブルのレコードを残すのか、右に書いたテーブルのレコードを残すのか、それが、左右の違いとなります。

\subsubsection{左外部結合}

まず、左外部結合について説明をします。以下のようなJOINを考えます。

\begin{verbatim}
SELECT name.animals , name.kinds FROM
  animals LEFT JOIN kinds ON kindid.animals = id.kinds ;
\end{verbatim}

LEFT JOINもしくはRIGHT JOINでは、ONのあとに書かれた条件に一致する組み合わせで作られたレコードは、そのまま結果のテーブルに残ります。これは、INNER JOINと同じです。

では、このJOINを、NULLを許容する形に拡張しましょう。
まず、全ての組み合わせの可能性のなかから、ON以降の条件が成立するものが残ります。

では、あるanimalsのレコードについて、ON以降の比較式が成立しないとき、それと組み合わされるべき、kinds由来のレコードはどれでしょうか。その答えは、適用できない、となります。なぜなら、左外部結合は、LEFT JOINの左側のレコードと、組み合わせるべきレコードがないことを許容するけつごうだからです。

このように、結合子の左に書いたテーブルのレコードに由来したレコードは、必ずJOIN後のテーブルに残ります。そのため、このような結合を、左外部結合といいます。

\subsubsection{右外部結合}

右外部結合は、左外部結合を逆にしたものです。結合子RIGHT JOINの右に書いたテーブルのレコードと一致するレコードが右に書いたテーブルにない場合、右側のテーブルのレコードと組み合わせでは、適用できるものがない、というレコードが残ります。

そのため、右側のテーブルのレコードが残る、ということから、右外部結合といいます。

\begin{verbatim}
SELECT name.animals , name.kinds FROM
  kinds RIGHT JOIN animals ON id.kinds = kindid.animals ;
\end{verbatim}

また、左右の外部結合は、置き換えることが可能です。上記のSELECT文は、左外部結合の説明に使ったものを、右外部結合で書き直したものになります。

\subsection{左右の外部結合と内部結合、交差結合の関係}

左右の外部結合と、内部結語の関係は何なのでしょうか。内部結合とは、NULLが出てこないときの左外部結合であり、右外部結合です。つまり、左右の外部結合の、特別な場合と考えることができます。

また、左右の外部結合は、内部結合の条件を、適用できない、判断できない場合について許容した、緩くしたものと考えることができます。
さらに、判定条件そのものをなくして、全ての可能性を取り込むこのが、交差結合です。


