\chapter{テーブルとレコードとアトリビュート}

SQLを学ぶために、まず、テーブルとレコードとアトリビュートは、どのようなものかを見ていくことにしましょう。前の章で、テーブルは丸、レコードはその丸の中のどこかにあるもの、という説明をしました。そして、アトリビューtについての説明ははしょりました。ぎt

この章では、この先SQLを学ぶのに必要なこととして、テーブルとレコードとアトリビュートについて、学び直していくことにします。


\section{テーブル}

リレーショナルデータベースのテーブルとは、どのようなものでしょうか。一言で言ってしまえば、レコードという要素がどこかに存在する空間です。レコードは、この空間の中のどこかにある要素となります。

\subsection{テーブルの中でレコードはどこにあるのか}

テーブルの中で、レコードはどこにあるのでしょうか。テーブルという空間の中で、レコードがどこにあるかは、そのレコードのアトリビュートの値で決まります。詳細は後で説明しますが、アトリビュートが座標と考えることができます。

たとえば、整数のアトリビュートを一つだけ持つレコードが定義されているとします。このとき、レコードを、数直線上のその値に対応している馬謖尾置いたと考えます。そうすると、テーブルという空間は数直線の形をしていて、レコードの場所は、その数直線上のどこかという様に考えることができます。


\subsection{テーブルの中身には順番がない}


テーブルの中にはレコードがあるわ出ですが、そのレコードにはなんらかの順番があるのでしょうか。結論から行けば、テーブルの中にあるレコードに、順番はありません。データベースに登録した順番や、特定のアトリビュートの値が順番になるのではありません。データベースのなかでは、レコードは順番が定義されません。

大切なことなのでもう一度言いますが、テーブルの中のレコードには、順番がありません。これは、読み出しを行ったとき、入れた順番や、プライマリキーの値の順番で結果が取り出せることを期待してはならない、ということです。
これは、limitを使ってひとつだけ要素を取り出す場合、特定のレコードが必ず取り出せることを期待s知恵はならないということでもあります。


では、ORDER BYで定義される順番はなんなのでしょうか。それは、データベースから取り出すときにソーティングを行って、定義された順番で並べています。そのため、順番で取り出す必要が無いときは、ORDER BYを書かない方が読み出しは早くなります。

\subsection{普通の麻雀と鷲巣麻雀}



\subsection{順番はどうやってつけるのか}

\section{レコードとアトリビュート}


\subsection{アトリビュートの役割と座標軸}

\subsection{テーブルという空間}

\subsection{多次元空間}