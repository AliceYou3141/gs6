\chapter{SQLの本をどう選ぶ?}

SQLを学ぼうとするとき、はず、参考書を探すというのはよくあることです。ですが、署名にSQLとついた本を書店で眺めたり、Amazon当たりのネット通販で眺めたりしたとき、どの穂Jンを手に取っていいか分からなくなることがあります。

本書では、最初に、SQLを学ぶための本の選択の仕方に土江記載します。

\section{sQLの本とRDBMSの本は違う}

たとえば、AmazonでSQLをキーワードに和書を検索したら、どのくらいの本が出てくるのでしょうか。2019年3月現在で、1000種類以上の津録があります。

この数では、SQLそのものを学ぶために、どの本を手に取ったらいいか分からなくなります。署名だけで買ってみると、SQLについては全くあっ枯れていない本を引き当ててしまうこともあります。

なぜこのような混迷したことになっているか、というと、SQLという単語が署名につく本には、データベースを操作する言語としてのSQLそのものについて書いてある本と、操作にSQLを使うことができる、リレーショナルデータベース実装の取り扱いについての本と、両方が含まれるためです。

このような混乱を招くのは、SQLを使用することができるリレーショナルデータベースの実装で、書籍が出るくらいメジャーなものは、なんとかSQL,もしくはSQLなんとか、という名前がついていることです。単独で書籍が出るものだと、Microsoftの製品のSQL-Serverや、Oracleに吸収されたMySQL ABのMySQL、PostgreSQLがあります。また、PHPなどに組み込まれている、ソフトウェア組み込み用のデータベースのSQL-Liteというものもあります。

もちろん、なんとかSQLでないリレーショナルデータベースもあります。有名なところでは、Oracle,DB2といった商用データベース、オープンソースだと、FireBirdや、MySQLからフォークした、MariaDBなどがあります。

\subsection{SQLの本}

\subsection{RDBMSの本}

\subsection{•}